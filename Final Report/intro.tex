\section{Introduction}
Being able to understand the similarity between words is a core component of many modern computational systems. From a search engine, which needs to understand that the user queries \textit{How tall is tom cruise?} and \textit{Tom cruise height?} have the same intent. To an e-commerce website that can use similarity between \textit{car parts} and \textit{auto parts} to provide similar product recommendations being able to recognize and understand similarity between words is crucial. Supervised methods tend to rely on carefully constructed synonym dictionaries or labeled datasets. Unsupervised methods like contextual \cite{Devlin2019BERTPO} and static word embeddings \cite{Mikolov2013DistributedRO}  provide a scalable and highly accurate notion of synonym search using vector similarity but once again rely on a large and diverse corpus for pre-training. Whether labeled or unlabeled, low-resource languages do not always have the luxury of large data sources. A learning methods that find alternative methods of dataset construction will prove useful to scale nlp systems. \\ \\
Modern commercial search engines have hundreds of millions of daily active users around the world. While users may speak different languages and search for different document, the use of search engines as a source of knowledge brings all users together. Every time a user issues a query and engages with a document they are providing a signal to the search engine. User clicks have long been used to improve information retrieval systems \cite{Chuklin2013ClickMI} but because of the sensitive nature of users queries, few datasets have been released publicly and there has been even less public research into these datasets. \\ \\
Building on the Question answering and Information Retrieval benchmark, MSMARCO \cite{Campos2016MSMA}, the ORCAS \cite{Craswell2020ORCAS2M} dataset features the largest publicly accessible document click dataset. It features 18.4 million document URL clicks on 10.4 million queries and 1.4 million documents which were extracted from the logs of a commercial search engine. This dataset captures both the similarity and difference of user searches. For example, the query \textit{pandas} features clicks on documents relating to the python programming library and the panda animal yet the query \textit{panda} features clicks on documents related to the restaurant chain panda express and the panda animal. \\ \\
Using the ORCAS dataset, we seek to explore a broad array of data mining techniques to produce novel training data to be used for transfer learning in tasks like query rewriting and synonym prediction. We will explore the usage dynamic network alignment, topic clustering and link prediction to produce training data which we will then use to explore its effects on the aforementioned tasks. Our goal is to explore how well click based data can replace traditional training corpora. \\