\section{Related Work}
We break our relevant reading into 3 sections: network mining, query rewriting, and synonym detection. 
\subsection{Network and Click Mining}
Network Mining has been extensively used in various domains such as social networks and biomedical domain for performing some downstream analytic tasks such as recommender systems, link prediction, node classification etc \cite{FournierViger2017ASO, Ghosh2012ATR}. One popular sub-branch of Network Mining is Click Mining that uses user clicks, co-clicks and query logs for mining interesting patterns. Since user click logs are private hence collection of such a dataset becomes sensitive therefore we do not have many publicly available datasets for this. \cite{Hua2013ClickageTB} introduce a new dataset  - Clickage - for click mining using user search logs. ORCAS \cite{Craswell2020ORCAS2M} dataset features the largest publicly accessible document click dataset.  Several papers are research of the area of Click Mining. \cite{Cao2008ContextawareQS} makes use of a user's immediate and preceding queries in the search query log for a context-aware query suggestion. Some others \cite{Li2013EnhancedIR, Chuklin2013ClickMI, Kacprzak2017AQL} also make use of Click Mining of Query logs for basic Information Retrieval tasks.

%\subsection{Transfer Learning and Distant Supervision}

\subsection{Query Rewriting}
Query rewriting has always been an important problem in information retrieval. One approach \cite{xu1996query} is to expand the query using the top ranked documents from the original query. Later approaches \cite{cui2002probabilistic, antonellis2008simrank++} rely on user generated data and focus on using user query logs to generate expansions via probabilistic frameworks. In constrast, we plan to leverage pre-trained language models to learn the distribution over queries with co-clicks and do query rewriting in a generative fashion. 
%\cite{Strohman2005IndriA}
%\cite{Cheriton2019FromDT}
%\cite{Jiang2016LearningQA}
%\cite{Radlinski2010InferringQI}
\subsection{Synonym Prediction}
HolisticOpt \cite{he2016automatic} aligns with our thought process as it makes use of query log clicks and web table attribute name co-occurrences to find attribute synonyms that can be used to boost the performance of search engines. \citeauthor{10.1145/3292500.3330914} propose a synonym prediction approach for the medical domain by using a multi-task model with a hierarchical term task relationship to learn entity embeddings.
%\cite{Ansari2020IdentifyingSD}
%\cite{Bona2010LearningDM}

Transfer learning is an active area of research in machine learning. Recent approaches have shown transfer learning via distant supervision to be effective in the domains of relation extraction \cite{Mintz2009DistantSF, Ji2017DistantSF}, multilingual models \cite{Hedderich2020TransferLA},  domain adaptation \cite{zhang2020multi}, question answering \cite{reddy2020end} and information retrieval \cite{Mitra2020NeuralMF}.








