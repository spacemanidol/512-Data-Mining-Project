\documentclass[sigplan,screen]{acmart}
%% NOTE that a single column version is required for 
%% submission and peer review. This can be done by changing
%% the \doucmentclass[...]{acmart} in this template to 
%% \documentclass[manuscript,screen,review]{acmart}
%% 
%% To ensure 100% compatibility, please check the white list of
%% approved LaTeX packages to be used with the Master Article Template at
%% https://www.acm.org/publications/taps/whitelist-of-latex-packages 
%% before creating your document. The white list page provides 
%% information on how to submit additional LaTeX packages for 
%% review and adoption.
%% Fonts used in the template cannot be substituted; margin 
%% adjustments are not allowed.
%%
%% \BibTeX command to typeset BibTeX logo in the docs
\AtBeginDocument{%
  \providecommand\BibTeX{{%
    \normalfont B\kern-0.5em{\scshape i\kern-0.25em b}\kern-0.8em\TeX}}}

\begin{document}
\title{CS512 Proposal: Leveraging document co-clicks to infer similarity}


\author{Daniel Campos} \author{Revanth Reddy}
\author{Shweta Garg} 
\email{dcampos3, revanth3, shwetag2 @illinois.edu}


%%
%% The abstract is a short summary of the work to be presented in the
%% article.
\begin{abstract}
In the last few decades search engines like Google, Bing, Baidu, and Yandex have become the primary way that people around the world interact with information. Their constant and diverse usage has made these engines idea sources of training data like document co-clicks. Using the ORCAS dataset we seek to explore the intersection of Data Mining and deep learning and answer the question: Can document co-clicks be used to learn similarity between concepts?
\end{abstract}

%%
%% The code below is generated by the tool at http://dl.acm.org/ccs.cfm.
%% Please copy and paste the code instead of the example below.
%%
\begin{CCSXML}
<ccs2012>
<concept>
<concept_id>10002951.10003317.10003325.10003330</concept_id>
<concept_desc>Information systems~Query reformulation</concept_desc>
<concept_significance>500</concept_significance>
</concept>
<concept>
<concept_id>10002951.10003317.10003325.10003328</concept_id>
<concept_desc>Information systems~Query log analysis</concept_desc>
<concept_significance>500</concept_significance>
</concept>
<concept>
<concept_id>10002951.10003227.10003351.10003446</concept_id>
<concept_desc>Information systems~Data stream mining</concept_desc>
<concept_significance>500</concept_significance>
</concept>
</ccs2012>
\end{CCSXML}

\ccsdesc[500]{Information systems~Query reformulation}
\ccsdesc[500]{Information systems~Query log analysis}
\ccsdesc[500]{Information systems~Data stream mining}

%%
%% Keywords. The author(s) should pick words that accurately describe
%% the work being presented. Separate the keywords with commas.
\keywords{datasets, neural networks, information retrieval, data mining}

%%
%% This command processes the author and affiliation and title
%% information and builds the first part of the formatted document.
\maketitle
\input{1description}
\section{Main Ideas}
The main focus of our work is to explore what kind of value we can extract out of the ORCAS dataset with regards to generating distant supervision data. We believe that if our efforts our successful this method can be explored for low resource languages.
\subsection{Dataset Mining}
Our first idea is focus on the general exploration and processing of the ORCAS dataset and the application of traditional data mining algorithms to explore what signals we can extract. We seek to create a set of signals which we can use in our downstream experiments to prove or disprove our hypothesis. Some of the methods we plan to explore include: phrase based clustering, document based clustering, n-gram association analysis, etc.
\subsection{query synonym prediction}
Once we have produced various dataset we will explore how these datasets can be used for transfer learning. Using contextual and non contextual word representations we will explore how well each of these methods can represent n-gram and query similarity. Evaluation will focus on predicting the degree of connection between queries(measured by proximity in click graph) and between n-gram terms.
\subsection{Query rewriting}
Similar to synonym detection, we will explore the effect of using our processed dataset on information retrieval centric domain. Using the MSMARCO baseline (other baselines?) we will explore how the use of our processed data can be applied to improve existing and established ranking models like BM25.  
\section{Preliminary Plan}
Our plan essentially has two stages: data exploration and transfer learning. 
\subsection{Milestones}
\begin{enumerate}
    \item Project Scoping(Feb 25th): Provide a scope of work to be attempted and problem framing.
    \item Data Exploration and Clustering(March 15th): Application of Data Mining algorithms to visualize and explore clusters both of query terms and documents.
    \item Transfer Learning Labels(March 28): using finding from data exploration and clustering we will create processed data to be used with our transfer tasks.
    \item Synonym Evaluation dataset and Baselines(March 28th):Finalize query synonym prediction task and produce baselines that do not leverage click data.
    \item Query Rewriting baseline(March 28): Finalize query rewriting task and produce baselines that do not leverage click data.
    \item Midterm Report(March 30): discuss progress and learning.
    \item Experiments across tasks(April 15th): Initial results using transfer data on benchmark tasks. Use results to go back and tweak mining methods.
    \item Experiments across tasks v2(April 30): Updated results using improved data.
    \item Final report (May 5th)
\end{enumerate}
\subsection{Roles}
\begin{itemize}
  \item \textbf{Daniel Campos}: Application of Data Mining techniques to dataset for exploration and transfer label creation, report writing, IR experiments
  \item \textbf{Revanth Reddy}: Query rewriting baseline, experiments and tweaking.
  \item \textbf{Shweta Garg}: Synonym predicting baseline, experiments and tweaking. 
\end{itemize}

\section{Related Work}
We break our relevant reading into 3 sections: network mining, query rewriting, and synonym detection. 
\subsection{Network and Click Mining}
Network Mining has been extensively used in various domains such as social networks and biomedical domain for performing some downstream analytic tasks such as recommender systems, link prediction, node classification etc \cite{FournierViger2017ASO, Ghosh2012ATR}. One popular sub-branch of Network Mining is Click Mining that uses user clicks, co-clicks and query logs for mining interesting patterns. Since user click logs are private hence collection of such a dataset becomes sensitive therefore we do not have many publicly available datasets for this. \cite{Hua2013ClickageTB} introduce a new dataset  - Clickage - for click mining using user search logs. ORCAS \cite{Craswell2020ORCAS2M} dataset features the largest publicly accessible document click dataset.  Several papers are research of the area of Click Mining. \cite{Cao2008ContextawareQS} makes use of a user's immediate and preceding queries in the search query log for a context-aware query suggestion. Some others \cite{Li2013EnhancedIR, Chuklin2013ClickMI, Kacprzak2017AQL} also make use of Click Mining of Query logs for basic Information Retrieval tasks.

%\subsection{Transfer Learning and Distant Supervision}

\subsection{Query Rewriting}
Query rewriting has always been an important problem in information retrieval. One approach \cite{xu1996query} is to expand the query using the top ranked documents from the original query. Later approaches \cite{cui2002probabilistic, antonellis2008simrank++} rely on user generated data and focus on using user query logs to generate expansions via probabilistic frameworks. In constrast, we plan to leverage pre-trained language models to learn the distribution over queries with co-clicks and do query rewriting in a generative fashion. 
%\cite{Strohman2005IndriA}
%\cite{Cheriton2019FromDT}
%\cite{Jiang2016LearningQA}
%\cite{Radlinski2010InferringQI}
\subsection{Synonym Prediction}
HolisticOpt \cite{he2016automatic} aligns with our thought process as it makes use of query log clicks and web table attribute name co-occurrences to find attribute synonyms that can be used to boost the performance of search engines. \citeauthor{10.1145/3292500.3330914} propose a synonym prediction approach for the medical domain by using a multi-task model with a hierarchical term task relationship to learn entity embeddings.
%\cite{Ansari2020IdentifyingSD}
%\cite{Bona2010LearningDM}

Transfer learning is an active area of research in machine learning. Recent approaches have shown transfer learning via distant supervision to be effective in the domains of relation extraction \cite{Mintz2009DistantSF, Ji2017DistantSF}, multilingual models \cite{Hedderich2020TransferLA},  domain adaptation \cite{zhang2020multi}, question answering \cite{reddy2020end} and information retrieval \cite{Mitra2020NeuralMF}.









\bibliographystyle{ACM-Reference-Format}
\bibliography{bibliography}
%\appendix
\end{document}
\endinput

